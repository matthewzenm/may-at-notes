\documentclass[a4paper]{amsart}

\usepackage{amsmath}
\usepackage{amssymb}
\usepackage{amsthm}
\usepackage{geometry}
\usepackage{tikz-cd}
\usepackage{mathrsfs}
\usepackage{hyperref}

\allowdisplaybreaks[4]

\theoremstyle{plain}
\newtheorem{thm}{Theorem}[section]
\newtheorem{prop}[thm]{Proposition}
\newtheorem{lem}[thm]{Lemma}
\newtheorem{cor}[thm]{Corollary}
\theoremstyle{definition}
\newtheorem*{defn}{Definition}
\newtheorem{eg}[thm]{Example}
\theoremstyle{remark}
\newtheorem{rem}[thm]{Remark}

\DeclareMathOperator*{\colim}{colim}
\DeclareMathOperator{\sk}{sk}

\title{Seminar 1: Fundamental Group}
\author{Zeng Mengchen}
\date{\today}

\begin{document}
\maketitle

\section{Fundamental Group}

\subsection{Definition}
We have learnt the notion of fundamental group in topology course, now we review it here.

A \emph{homotopy} between maps $f,g:X\to Y$ is a map $h:I\times X\to Y$ such that $h(0,x)=f(x)$, $h(1,x)=g(x)$.
Two paths $f,g:a\to b$ in $X$ is defined to be equivalent if there exists a homotopy $h:I\times I\to X$ such that
\[h(t,0)=a,\ h(t,1)=b,\ h(0,s)=f(s),\ h(1,s)=g(s).\]
Denote the equivalence class of $f$ by $[f]$.
A \emph{loop} is a path with $f(0)=f(1)$.

Define $\pi_1(X,x)$ to be the collection of all equivalent class of loops with starting point and end point $x$.
We give a group structure on $\pi_1(X,x)$.
Let $f:a\to b$, $g:b\to c$ be paths, define $g\cdot f$ to be the the path
\[g\cdot f(s)=\begin{cases}
    f(2s), & 0\leq s\leq 1/2,\\
    g(2s-1), & 1/2\leq s\leq 1,
\end{cases}\]
and define $[g][f]:=[g\cdot f]$.
We also define the inverse $g^{-1}(s)=g(1-s)$ and $[g]^{-1}=[g^{-1}]$.
Moreover, the identity is defined to be the constant loop $c_x$.
Clearly these definitions make $\pi_1(X,x)$ into a group.

\subsection{Choice of Base Point}
Let $a:x\to y$ be a path, we define a homomorphism from $\gamma[a]:\pi_1(X,x)\to\pi_1(X,y)
$ to be $\gamma[a][f]=[a\cdot f\cdot a^{-1}]$.
Clearly we have $\gamma[b\cdot a]=\gamma[b]\circ\gamma[a]$, hence $\gamma[a]$ is an isomorphism.
Therefore, we have
\begin{prop}
    For $x,y\in X$ in the same path connected component, $\pi_1(X,x)$ is (not necessarily canonical) isomorphic to $\pi_1(X,y)$.
\end{prop}

If $\pi_1(X,x)$ happens to be abelian, then let $a':x\to y$ be another path, we have
\begin{align*}
    \varphi[(a')^{-1}]\circ\varphi[a]([f])&=[(a')^{-1}\cdot a][f][a^{-1}\cdot a']\\
    &=[f][(a')^{-1}\cdot a][a^{-1}\cdot a']\\
    &=[f],
\end{align*}
that is, the isomorphism between $\pi_1(X,x)$ and $\pi_1(X,y)$ does not depend on the choice of path, namely, canonical.

\subsection{Homotopy Invariance}

Let $p:X\to Y$ be a map, define $p_*:\pi_1(X,x)\to\pi_1(Y,p(x))$ to be
\[p_*[f]=[p\circ f].\]
Clearly $p_*$ is a homomorphism, and identity map $\operatorname{id}:X\to X$ induces the identity homomorphism.
Moreover, if $p:X\to Y$, $q:Y\to Z$, then we have $q_*\circ p_*=(q\circ p)_*$.

Now suppose given two maps $p,q:X\to Y$ and a homotopy $h:p\simeq q$, a path $a(t)=h(x,t)$, we have
\begin{prop}
    The following diagram is commutative:
    \[\begin{tikzcd}
         & \pi_1(X,x) \ar[dl, "{p_*}"'] \ar[dr, "{q_*}"] & \\
        \pi_1(Y,p(x)) \ar[rr, "{\gamma[a]}"] & & \pi_1(Y,q(x)).
    \end{tikzcd}\]
\end{prop}
\begin{proof}
    Let $f:I\to X$ be a loop, we must show $q\circ f$ is equivalent to $a\cdot(p\circ f)\cdot a^{-1}$.
    This is equivalent to show the constant loop $c_{p(x)}$ is equivalent to $a^{-1}\cdot(q\circ f)^{-1}\cdot a\cdot(p\circ f)$.
    Define $j:I\times I\to Y$ by $j(t,s)=h(t,f(s))$, then
    \[j(0,s)=(p\circ f)(s),\ j(1,s)=(q\circ f)(s),\ j(t,0)=a(t)=j(t,1).\]
    Thus we have a schematic diagram of $j$:
    \[\begin{tikzcd}[sep=huge]
        \ \ar[r, "a"] & \ \\
        \ \ar[u, "{p\circ f}"] \ar[r, "a"'] & \ \ar[u, "{q\circ f}"']
    \end{tikzcd}\]
    Thus, going clockwise around the boundary starting at $(0,0)$, we traverse $a^{-1}\cdot(q\circ f)^{-1}\cdot a\cdot(p\circ f)$.
    Since linear homotopy connects the boundary to origin, $j$ induces a homotopy between the composite and $c_{p(x)}$.
\end{proof}

\subsection{Calculations}
\begin{lem}
    $\pi_1(\mathbb{R},0)=0$.
\end{lem}
\begin{proof}
    We use linear homotopy $k:I\times\mathbb{R}\to\mathbb{R}$ by $k(t,s)=(1-t)s$, then $k$ connects identity and constant map at $0$.
    Hence by homotopy invariance, we have $\pi_1(\mathbb{R},0)=0$.
\end{proof}

\begin{prop}
    $\pi_1(\mathbb{S}^1,1)\cong\mathbb{Z}$.
\end{prop}
Here we regard $\mathbb{S}^1=\{z\in\mathbb{C}:\ |z|=1\}$.
We postpone the proof until we develop the theory of covering spaces.

\section{Catagorical Language}

\subsection{Categories}
A \emph{category} consists of a collection of objects and a set $\hom(A,B)$ of morphisms for each pair of objects $A$ and $B$.
In each $\hom(A,B)$, there exists an identity morphism $\operatorname{id}$ and a composition law
\[\circ:\hom(B,C)\times\hom(A,B)\to\hom(A,C)\]
for each triple of objects $A,B,C$.
Composition and identity morphisms must make the following diagrams commute:
\[\begin{tikzcd}
    A \ar[r, "f"] \ar[dr, "{g\circ f}"'] & B \ar[d, "g"] \ar[dr, "{h\circ g}"] \\
     & C \ar[r, "h"] & D,
\end{tikzcd}\]
\[\begin{tikzcd}
    A \ar[r, "f"] \ar[dr, "{f}"'] & B \ar[d, "{\operatorname{id}}"] \\
     & B,
\end{tikzcd}\]
\[\begin{tikzcd}
    A \ar[d,"{\operatorname{id}}"'] \ar[dr,"f"] & \\
    A \ar[r, "f"'] & B.
\end{tikzcd}\]
A morphism $f\in\hom(A,B)$ is called an \emph{isomorphism} if there exists a $g\in\hom(B,A)$ such that $g\circ f=\operatorname{id}_A$, $f\circ g=\operatorname{id}_B$.

We list some frequently used categories here.
\begin{itemize}
    \item $\mathsf{Set}$, category of sets, with morphisms to be mappings.
    \item $\mathsf{Grpd}$, category of small groupoids.
    Small groupoids are defined to be small categories (i.e.\ whose collection of objects is a set) with all morphisms are isomorphisms, and morphisms between groupoids are functors (whose definition will be expalined later).
    \item $\mathsf{Grp}$, category of groups.
    Groups are groupoids with single object.
    \item $\mathsf{Ab}$, category of abelian groups.
    \item $\mathsf{Top}$, category of topological spaces, with morphisms to be continuous mappings, i.e.\ maps.
    \item $\mathsf{Top}_*$, category of pointed topological spaces, with morphisms to be maps that map the base point to base point.
\end{itemize}

We also define the \emph{opposite category} of a category $\mathsf{C}$ (denoting $\mathsf{C}^{\mathrm{op}}$) consisting of same objects as $\mathsf{C}$ and $\hom_{\mathsf{C}^{\mathrm{op}}}(A,B)=\hom_\mathsf{C}(B,A)$.

\subsection{Functors and Natural Transformations}

We define a \emph{covariant functor} $F$ from category $\mathsf{C}$ to category $\mathsf{D}$ to be an assignment of each object $A$ in $\mathsf{C}$ an object $F(A)$ in $\mathsf{D}$, and of each morphism $f$ in $\hom(A,B)$ a morphism $F(f)\in\hom(F(A),F(B))$.
The functor $F:\mathsf{C}\to\mathsf{D}$ should be distributive to composition and preserve identity, that is, $F(g\circ f)=F(f)\circ F(g)$ for any $f\in\hom(A,B)$, $g\in\hom(B,C)$ and $F(\operatorname{id}_A)=\operatorname{id}_{F(A)}$.
Also we define a \emph{contravariant functor} from $\mathsf{C}$ to $\mathsf{D}$ to be a covariant functor from $\mathsf{C}^{\mathrm{op}}$ to $\mathsf{D}$.

A \emph{natural transformation} from a functor $F:\mathsf{C}\to\mathsf{D}$ to a functor $G$ is a family of morphisms $\varphi_A:F(A)\to G(A)$ making the following diagram commute for any object $A,B$ and $f\in\hom(A,B)$
\[\begin{tikzcd}
    F(A) \ar[r, "{\varphi_A}"] \ar[d, "F(f)"] & G(A) \ar[d, "{G(f)}"] \\
    F(B) \ar[r, "{\varphi_B}"] & G(B).
\end{tikzcd}\]

\begin{eg}
    \begin{enumerate}
        \item The forgetful functor $\mathsf{Grp}\to\mathsf{Set}$, which maps a group to its underlying set.
        \item The free group functor $\mathsf{Set}\to\mathsf{Grp}$, which maps a set $S$ to a free group $F(S)$ with basis $S$.
        A free group with basis $S$ is characterized by the following universal property:
        any mapping from $S$ to a group $G$ factors through the natural embedding $\iota:S\to F(S)$, that is for $f:S\to G$, there exists a unique homomorphism $\tilde{f}$ making the following diagram commute:
        \[\begin{tikzcd}
            F(S) \ar[r, dashed, "\tilde{f}"] & G \\
            S \ar[u, "\iota"] \ar[ur, "f"].
        \end{tikzcd}\]
        If $g:S\to T$ is a mapping of sets, then $F(g)$ is induced by above universal property.
        \item The natural embedding of double dual in the category of vector spaces.
        Let $\operatorname{Id}$ be the identity functor, and denote the double dual of $V$ by $V^{**}$.
        Then $v\mapsto(f\mapsto f(v))$ for any $v\in V$ and linear functional $f$ defines an injection from $V$ to $V^{**}$.
        One can check the family of injections consists a natural transformation from $\operatorname{id}$ to double dual.
        In particular, if $V$ is finite dimensional, this shows $V^{**}$ is canonically isomorphic to $V$.
    \end{enumerate}
\end{eg}

\subsection{Colimits and Limits}
In category theory, an object $A$ being initial means for any other objects $B$, $\hom(A,B)$ is a singleton.
Being terminal reverses the morphisms, this means $\hom(B,A)$ is a singleton.
A simple exercise is to show that initial objects and terminal objects are unique up to an isomorphism if they exist.

Let $\mathsf{I}$ be a small category.
An $\mathsf{I}$-shaped diagram in category $\mathsf{C}$ is a functor $F:\mathsf{I}\to\mathsf{C}$.
A morphism between $\mathsf{I}$-shaped diagrams is a natural transformation.
Let $A$ be an object in $\mathsf{C}$, then there is a functor that maps all objects in $\mathsf{I}$ to $A$ and all morphisms to $\operatorname{id}_A$.
Denote this $\mathsf{I}$-diagram by $\underline{A}$.

Let $F$ be an $\mathsf{I}$-diagram, the \emph{colimit} of $F$ is the initial object among all morphisms having the form $\eta:F\to\underline{A}$, that is, $\eta$ factors through $\iota:F\to\underline{\colim{F}}$.
Conversely, the \emph{limit} of $F$ is the terminal object among all such morphisms, that is, $\varepsilon:\underline{A}\to F$ factors through $\pi:\underline{\lim{F}}\to F$.
Expressing in diagrams, for each map $f:D\to D'$ in $\mathsf{I}$, we have the commutative diagram for colimit:
\[\begin{tikzcd}
    F(D)\ar[rr, "{F(f)}"] \ar[dr, "\iota"] \ar[ddr, "\eta"] & & F(D') \ar[dl, "\iota"] \ar[ddl, "\eta"] \\
     & \colim{F} \ar[d, dashed, "{\tilde\eta}"] & \\
     & A, &
\end{tikzcd}\]
and the commutative diagram for limit:
\[\begin{tikzcd}
    F(D) \ar[rr, "{F(f)}"] & & F(D')\\
     & \lim{F} \ar[lu, "\pi"] \ar[ru, "\pi"] & \\
     & A. \ar[luu, "\varepsilon"] \ar[u, dashed, "{\tilde\varepsilon}"] \ar[ruu, "\varepsilon"]
\end{tikzcd}\]

Practically, if the diagram $F$ consists of small category $\mathsf{U}$, we usually write
\[\colim{F}=\colim_{U\in\mathsf{U}}U,\]
and if $\mathsf{I}$ serves as an index set of $\{U_i\}_{i\in\mathsf{I}}$, we also write
\[\colim{F}=\colim_{i\in\mathsf{I}}U_i.\]
For limits there are also similar notations.

\begin{eg}We list some notions here.
    \begin{enumerate}
        \item Products and coproducts.
        If $\mathsf{I}$ is a discrete category, then the limits and colimits indexed on $\mathsf{I}$ are products and coproducts.
        The products in $\mathsf{Set}$, $\mathsf{Top}$, $\mathsf{Grp}$ are all Cartesian products with morphisms are projections.
        The coproducts in $\mathsf{Set}$ and $\mathsf{Top}$ are disjoint unions, and coproducts in $\mathsf{Grp}$ are free products (multiplications are adjunctions).
        \item Pushouts and pullbacks.
        This is the colimit of the diagram
        \[\begin{tikzcd}
            e & d \ar[l] \ar[r] & f,
        \end{tikzcd}\]
        and the limit of the diagram
        \[\begin{tikzcd}
            e \ar[r] & d & f, \ar[l]
        \end{tikzcd}\]
        respectively.
        \item Equalizers and coequalizers.
        This is the limit and colimit of the following diagram
        \[\begin{tikzcd}
            d \ar[r, shift left] \ar[r, shift right] & d'.
        \end{tikzcd}\]
    \end{enumerate}
\end{eg}

A category is said to be \emph{complete} if it has all limits, and to be \emph{cocomplete} if it has all colimits.
\begin{prop}
    A category is cocomplete if and only if it has coproducts and coequalizers.
\end{prop}
\begin{proof}
    One side is trivial, we prove that a category which has coproducts and coequalizers is cocomplete.
    We will abuse a lot of notation in the following proof.

    Let $\{U_i\}_{i\in\mathsf{I}}$ be indexed by $\mathsf{I}$.
    For any morphism $f:U_i\to U_j$, we compose $U_i\xrightarrow{f}U_j\xrightarrow{\iota}\bigsqcup_{j\in\mathsf{I}}U_j$, and obtain
    \[\bigsqcup_{f:U_i\to U_j}U_i\xrightarrow{\varphi}\bigsqcup_{j\in\mathsf{I}}U_j.\]
    Similarly, we obtain $\operatorname{id}$ between these two coproducts.
    Thus we have the coequalizer
    \[\begin{tikzcd}
        \bigsqcup_{f:U_i\to U_j}U_i \ar[rr, shift left, "\varphi"] \ar[rr, shift right, "{\operatorname{id}}"'] \ar[dr, "\eta"'] & & \bigsqcup_{j\in\mathsf{I}}U_j \ar[dl, "\delta"]\\
         & U. &
    \end{tikzcd}\]
    We claim $U$ is the colimit of $\{U_i\}_{i\in\mathsf{I}}$.
    We must verify the universal property.
    Suppose given a family of morphisms $f_i:U_i\to A$ compatible with morphisms in $\mathsf{I}$, we check arbitrary commutative diagram
    \[\begin{tikzcd}
        U_i \ar[rr, "f"] \ar[dr, "{f_i}"'] & & U_j \ar[dl, "{f_j}"] \\
         & A. &
    \end{tikzcd}\]
    Let $\iota_i:U_i\to\bigsqcup_{f:U_i\to U_j}U_i$ embeds $U_i$ to the one corresponds to $f$, $\iota_j:U_j\to\bigsqcup_{j\in\mathsf{I}}U_j$ be the natural embedding.
    Then $f_j$ factors through $\iota_j$, obtaining $f_j=\delta'\circ\iota_j$.
    For $U_i$ corresponds to $f$, we have $f_j\circ f=f_i\circ\operatorname{id}$, by the universal property, this means $\delta'\circ\varphi=\delta'\circ\operatorname{id}$.
    Thus we can define $\eta'=\delta'\circ\varphi$.
    Now we have a big commutative diagram:
    \[\begin{tikzcd}
        U_i \ar[rr, "f"] \ar[d, "{\iota_i}"] & & U_j \ar[d, "{\iota_j}"]\\
        \bigsqcup_{f:U_i\to U_j} U_i \ar[rr, shift left, "\varphi"] \ar[rr, shift right, "\operatorname{id}"'] \ar[dr, "\eta"'] \ar[ddr, "{\eta'}"'] & & \bigsqcup_{j\in\mathsf{I}}U_j \ar[dl, "\delta"] \ar[ddl, "\delta'"] \\
         & U \ar[d, dashed] & \\
         & A, &
    \end{tikzcd}\]
    the last thing we need to do is to show $f_i=\gamma_i\circ\iota_i$, but this follows directly from the commutativity of whole diagram.
\end{proof}

\begin{cor}
    $\mathsf{Grp}$ is cocomplete.
\end{cor}
\begin{proof}
    We show that $\mathsf{Grp}$ verifies coequalizers.
    Let $\varphi:G\to H$ and $\psi:G\to H$ be two homomorphisms, set
    \[K=H/\langle\varphi(g)\sim\psi(g)|\ g\in G\rangle.\]
    It's clear that $K$ satisfies the universal property.
\end{proof}

\begin{cor}
    The pushout in $\mathsf{Grp}$ 
    $\begin{tikzcd}
        H & G \ar[l,"\varphi"'] \ar[r, "\psi"] & K
    \end{tikzcd}$
    is given explicitly by $H*K$ quotient out the normal subgroup generated by $\varphi(g)\psi(g^{-1})$ for all $g\in G$.
\end{cor}
\begin{proof}
    By the construction of above proposition, the pushout is the coequalizer of
    \[\begin{tikzcd}
        G*H*K*G*G \ar[r,shift left, "f"] \ar[r,shift right, "\operatorname{id}"'] & G*H*K.
    \end{tikzcd}\]
    Then by above corollary, the coequalizer must be
    \begin{align*}
        G*H*K/\langle g_1hk\varphi(g_2)\psi(g_3)\sim g_1hkg_2g_3\rangle&\cong G*H*K/\langle\varphi(g_2)\psi(g_3)\sim g_2g_3\rangle\\
        &\cong G*H*K/\langle\varphi(g)\sim\psi(g)\sim g\rangle\\
        &\cong H*K/\langle\varphi(g)\sim\psi(g)\rangle.\qedhere
    \end{align*}
\end{proof}

By using opposite category formally, we can obtain
\begin{prop}
    A category is complete if and only if it has products and equalizers.
\end{prop}

\subsection{Group Object}

Let $\mathsf{C}$ be a category which has products and terminal object $*$.
A \emph{group object} consists of an object $G$ and a morphism $\mu:G\times G\to G$ called multiplication, a morphism $e:*\to G$ called unit element and a morphism $i:G\to G$ called inverse, satisfying\\
\emph{Associative law}
\[\begin{tikzcd}
    G\times G\times G \ar[r, "{\mu\times\operatorname{id}}"] \ar[d, "{\operatorname{id}\times\mu}"'] & G\times G \ar[d, "\mu"] \\
    G\times G \ar[r, "\mu"] & G.
\end{tikzcd}\]
\noindent\emph{Identity law}
\[\begin{tikzcd}
     & G \ar[dl, "\operatorname{id}\times e"'] \ar[d, "\operatorname{id}"] \ar[dr, "{e\times\operatorname{id}}"] & \\
    G\times G \ar[r, "\mu"] & G & G\times G. \ar[l, "\mu"']
\end{tikzcd}\]
\noindent\emph{Inverse}
\[\begin{tikzcd}
    & G \ar[dl, "\operatorname{id}\times i"'] \ar[d, "e"] \ar[dr, "{i\times\operatorname{id}}"] & \\
    G\times G \ar[r, "\mu"] & G & G\times G. \ar[l, "\mu"']
\end{tikzcd}\]
where $e$ is the composition $G\to * \to G$.

\begin{lem}
    If a functor preserves product, then it preserves group objects.
\end{lem}

Now we study topological groups.
Topological groups with base point $e$ are group objects in category $\mathsf{Top}_*$, that is, $G$ is a topological group if it is a topological space and multiplication and inverse are continuous.
We want to show that
\begin{prop}\label{top group}
    The fundamental group of a topological group (with base point $e$) is abelian.
\end{prop}

This proposition can be deduced by given direct homotopies, but we will adopt some ``abstract nonsense'' here.

\begin{lem}
    Fundamental group functor $\pi_1:\mathsf{Top}_*\to\mathsf{Grp}$ preserves product.
\end{lem}
\begin{proof}
    We will check the universal properties elaborately.
    Let $\{X_\alpha\}_{\alpha\in A}$ be topological spaces, $(X,p_\alpha)$ be their product.
    Let $G$ be a group, together with a family of homomorphisms $f_\alpha:G\to\pi_1(X_\alpha,x)$.
    Choose an element $g\in G$, suppose $f_\alpha(g)=[\gamma_\alpha]\in\pi_1(X_\alpha,x)$, $\alpha\in A$.
    Then $\gamma_\alpha:I\to X_\alpha$, by universal property, we have a $\gamma:I\to X$ such that $p_\alpha\circ\gamma=\gamma_\alpha$.
    Thus we define $f:G\to\pi_1(X,x)$ by $f(g)=[\gamma]$, then $p_{\alpha*}[\gamma]=[\gamma_\alpha]$.
    It's clear that such construction must be unique, hence $\pi_1(X,x)$ satisfies the universal property.
    Therefore
    \[\pi_1\left(\prod_{\alpha\in A}X_\alpha,x\right)=\prod_{\alpha\in A}\pi_1(X_\alpha,x).\qedhere\]
\end{proof}

\begin{lem}
    The group objects in $\mathsf{Grp}$ are abelian groups.
\end{lem}
\begin{proof}
    The inverse is defined by $g\mapsto g^{-1}$, but it is a homomorphism, we have $(gh)^{-1}=g^{-1}h^{-1}=h^{-1}g^{-1}$, hence the group object is abelian.
\end{proof}

Now the proof can be reduced into one sentence.
\begin{proof}[Proof of Proposition~\ref{top group}]
    Since fundamental group functor respects products, it maps group objects to group objects, that is, topological groups to abelian groups.
\end{proof}

\subsection{Categorical Equivalence and Skeleton}

We introduce the notion of two categories being equivalent.
Two categories $\mathsf{C}$ and $\mathsf{D}$ are \emph{equivalent} if there are functors $F:\mathsf{C}\to\mathsf{D}$ and $G:\mathsf{D}\to\mathsf{C}$ and natural isomorphisms $G\circ F\to\operatorname{id}_\mathsf{C}$ and $F\circ G\to\operatorname{id}_\mathsf{D}$.

Also we introduce the notion of a \emph{skeleton} $\sk\mathsf{C}$ of a category $\mathsf{C}$.
This is a ``full'' subcategory with one object from each isomorphism class of objects of $\mathsf{C}$.
``Full'' means that the morphisms between two objects of $\sk\mathsf{C}$ are the morphisms between all the objects in two isomorphism classes.

$\sk\mathsf{C}$ is equivalent to $\mathsf{C}$.
The inclusion functor $J:\sk\mathsf{C}\to\mathsf{C}$ gives the equivalence.
An inverse functor $F:\mathsf{C}\to\sk\mathsf{C}$ is obtained by letting $F(A)$ be the unique object in $\sk\mathsf{C}$ that is isomorphic to $A$, choosing an isomorphism $\alpha_A:A\to F(A)$ and defining $F(f)=\alpha_B\circ f\circ \alpha_A^{-1}:F(A)\to F(B)$ for a morphism $f\in\hom(A,B)$.
We choose $\alpha_A$ to be identity if $A$ is in $\sk\mathsf{C}$, and then $F\circ J=\operatorname{id}$, and by definition $\alpha_A$'s specify a natural isomorphism $\alpha:\operatorname{id}\to J\circ F$.

\section{Fundamental Groupoids}
Recall that a \emph{groupoid} is a category whose morphisms are all isomorphisms.
Let morphisms between groupoids be functors, we have the category $\mathsf{Grpd}$ of groupoids.

The \emph{fundamental groupoid} is a functor $\Pi:\mathsf{Top}\to\mathsf{Grpd}$ that maps a space $X$ to all the equivalence classes of paths in $X$.

We mention here the set of automorphisms of a point $x\in X$ is exactly the fundamental group $\pi_1(X,x)$.
Therefore, we have
\begin{prop}
    Let $X$ be a path connected space.
    For each point $x\in X$, the inclusion $\pi_1(X,x)\to\Pi(X)$ is an equivalence of categories.
\end{prop}
\begin{proof}
    We regard $\pi_1(X,x)$ as a category with a single object $x$, and it is a skeleton of $\Pi(X)$.
    To show this, we notice that any $y\in X$ can be connected with $x$ by a path $\gamma$, and $[\gamma]$ is an isomorphism in $\Pi(X)$.
\end{proof}

\section{The van Kampen Theorem}

\subsection{Statement of the Theorem}

We state the \textsc{van Kampen} theorem as follows.
\begin{thm}[van Kampen]\label{van Kampen}
    Let $X$ be a path connected space, choose a base point $x\in X$.
    Let $\mathscr{O}$ be a cover of $X$ by path connected open subsets such that the intersection of finite many subsets in $\mathscr{O}$ is still in $\mathscr{O}$, and $x$ is in each $U\in\mathscr{O}$.
    Regard $\mathscr{O}$ as a category whose morphisms are inclusions of subsets, and observe that the functor $\pi_1(\cdot,x)$ restricted to the space and maps in $\mathscr{O}$, gives a diagram
    \[\pi_1|_\mathscr{O}:\mathscr{O}\to\mathsf{Grp}\]
    of groups.
    Then we have
    \[\pi_1(X,x)\cong\colim_{U\in\mathscr{O}}\pi_1(U,x).\] 
\end{thm}

A useful special case of van Kampen theorem is that the space is covered by two open subsets.
\begin{cor}
    Let $X=U\cup V$ with $U,V$ open and path connected, $x\in U\cap V$.
    Then $\pi_1(X,x)$ is the pushout of the following diagram
    \[\begin{tikzcd}
        \pi_1(U\cap V,x) \ar[r] \ar[d] & \pi_1(U,x) \ar[d, dashed] \\
        \pi_1(V,x) \ar[r, dashed] & \pi_1(X,x).
    \end{tikzcd}\]
\end{cor}

\subsection{Examples of the van Kampen Theorem}

We give some examples of using van Kampen theorem to calculate fundamental group.
We now drop the notation for the base point, writing $\pi_1(X)$ instead of $\pi_1(X,x)$.

\begin{prop}
    Let $X=\bigvee_{i\in I}X_i$ with $X_i$'s being path connected, and each of $X_i$ contains a contractible neighborhood $V_i$ for its base point.
    Then $\pi_1(X)$ is the coproduct (i.e.\ free product) of groups $\pi_1(X_i)$.
\end{prop}
\begin{proof}
    Let $U_i$ be the union of $X_i$ and $V_j$ for all $j\neq i$.
    Take $\mathscr{O}$ be the $U_i$ and their finite intersections, and apply van Kampen theorem to $\mathscr{O}$.
    Since any intersection of two or more of the $U_i$ is contractible, the intersections make no contribution to the colimit, and the colimit must be the coproduct.
\end{proof}

\begin{cor}\label{wedge circle}
    The fundamental group of a wedge sum of circles is a free group with one generator for each circle.
\end{cor}

We now give two examples of computing the fundamental group of compact surfaces.
By the classification theorem of compact surfaces, any compact surface is homeomorphic to a sphere, or to a connected sum of tori $\mathbb{T}^2:=\mathbb{S}^1\times\mathbb{S}^1$, or to a connect sum of projective planes $\mathbb{RP}^2=\mathbb{S}^2/\mathbb{Z}_2$.
We will see shortly that $\pi_1(\mathbb{RP}^2)=\mathbb{Z}_2$.
Using this fact, we can compute the fundamental group of all compact surfaces by induction.
We will illustrate by two examples.

\begin{eg}
    We compute the fundamental group of two-holed torus $\mathbb{T}^2\#\mathbb{T}^2$.
    We decompose the computation into several steps.

    \emph{Step 1}. $\pi_1(\mathbb{T}^2-\{\mathrm{pt}\})\cong\mathbb{Z}*\mathbb{Z}$.
    Linear homotopy shows $\mathbb{T}^2-\{\mathrm{pt}\}\simeq\mathbb{S}^1\vee\mathbb{S}^1$, the conclusion follows from Corollary~\ref{wedge circle}.

    \emph{Step 2}. Let $A\subset\mathbb{T}^2-\mathbb{D}^2$ be an annulus with inner boundary being $\partial\mathbb{D}^2$, $A'$ also be an annulus on another punched torus chosen similarly, then $\mathbb{T}^2\#\mathbb{T}^2\cong(\mathbb{T}^2-\mathbb{D}^2)\sqcup_f(\mathbb{T}^2-\mathbb{D}^2)$ with a suitable $f:A\to A'$.
    We compute $\pi_1(A)\hookrightarrow\pi_1(\mathbb{T}^2-\mathbb{D}^2)$.
    First, we notice that $\mathbb{T}^2-\mathbb{D}^2\simeq\mathbb{T}^2-\{\mathrm{pt}\}$.
    Moreover, since annulus is homeomorphic to $\mathbb{S}^1\times I$, a loop in $I$ does not contribute to $\pi_1(A)$, hence we check a loop in the outer boundary of $A$.
    Let $a,b$ be the two generators of each circle in $\mathbb{S}^1\vee\mathbb{S}^1$, then the linear homotopy sends a loop with $n$ rounds to $(aba^{-1}b^{-1})^n$ (this can be shown by the polyhedral representation of $\mathbb{T}^2$).
    Thus we have
    \begin{align*}
        \pi_1(A)\cong\mathbb{Z}&\hookrightarrow\pi_1(\mathbb{T}^2-\mathbb{D}^2)\cong\mathbb{Z}*\mathbb{Z}\\
        n&\mapsto(aba^{-1}b^{-1})^n.
    \end{align*}

    \emph{Step 3}. Apply van Kampen theorem to the cover $\{\mathbb{T}^2-\mathbb{D}^2,\mathbb{T}^2-\mathbb{D}^2,A\}$, we have $\pi_1(\mathbb{T}^2\#\mathbb{T}^2)$ is the pushout
    \[\begin{tikzcd}
        \pi_1(A) \ar[r] \ar[d] & \pi_1(\mathbb{T}^2-\mathbb{D}^2)\\
        \pi_1(\mathbb{T}^2-\mathbb{D}^2). &
    \end{tikzcd}\]
    Let the fundamental group of two punched torus be $\langle a_1,b_1\rangle$ and $\langle a_2,b_2\rangle$ respectively, then we have
    \[\pi_1(\mathbb{T}^2\#\mathbb{T}^2)\cong\langle a_1,b_1,a_2,b_2|\ a_1b_1a_1^{-1}b_1^{-1}a_2b_2a_2^{-1}b_2^{-1}\rangle,\]
    here the second inclusion reverses the orientation.
\end{eg}

\begin{eg}
    The same process can be applied to Klein bottle $\mathbb{RP}^2\#\mathbb{RP}^2$.
    The key points are $\mathbb{RP}^2-\{\mathrm{pt}\}\simeq\mathbb{S}^1$,
    \begin{align*}
        \pi_1(A)\cong\mathbb{Z}&\hookrightarrow\pi_1(\mathbb{RP}^2-\mathbb{D}^2)\cong\mathbb{Z}\\
        n&\mapsto a^{2n},
    \end{align*}
    and hence $\pi_1(\mathbb{RP}^2\#\mathbb{RP}^2)\cong\langle a,b|\ a^2=b^2\rangle$.
    Notice that this group is isomorphic to $\langle a,b|\ abab^{-1}\rangle$, which is computed by the polyhedral representation of Klein bottle.
\end{eg}

\section{Proof of van Kampen Theorem}

We provide the proof of van Kampen theorem in this section.
We first state the groupoid version of van Kampen theorem.

\begin{thm}[van Kampen]
    Let $\mathscr{O}$ be a cover of a space $X$ by path connected open subsets such that the intersection of finite many subsets in $\mathscr{O}$ is again in $\mathscr{O}$.
    Regard $\mathscr{O}$ as a category whose morphisms are the inclusions of subsets and observe that the functor $\Pi$, restricted to the space and maps in $\mathscr{O}$, gives a diagram
    \[\Pi|_\mathscr{O}:\mathscr{O}\to\mathsf{Grpd}\]
    of groupoids.
    Then we have
    \[\Pi(X)\cong\colim_{U\in\mathscr{U}}\Pi(U).\]
\end{thm}
\begin{proof}
    We must verify the universal property.
    Let $\mathsf{C}$ be a groupoid, and a morphism $\eta:\Pi|_\mathscr{O}\to\underline{\mathsf{C}}$ of $\mathscr{O}$-shaped diagrams of groupoids, we must construct a morphism, i.e.\ a functor $\tilde\eta:\Pi(X)\to\mathsf{C}$ of groupoids that restricts to $\eta_U$ on $\Pi(U)$ for each $U\in\mathscr{O}$.

    On objects, that is, the points of $X$, we must define $\tilde\eta(x)=\eta_U(x)$ for $x\in U$.
    This is independent of the choice of $U$ since $\mathscr{O}$ is closed under finite intersections.

    We now consider morphisms, that is, the homotopy classes of paths in $X$.
    If $f:x\to y$ lies entirely in a particular $U$, then we must define $\tilde\eta([f])=\eta_U([f])$.
    Again, since $\mathscr{O}$ is closed under finite intersections, this specification is independent of the choice of $U$ if $f$ lies entirely in more than one $U$.
    By compactness, any path $f$ is the composite of finite many paths $f_i$, each of which lies entirely in single $U$.
    Then we define $\tilde\eta([f])$ to be the composite of $\tilde\eta([f_i])$'s.
    We must show $\tilde\eta$ is well-defined.
    Suppose $h:f\simeq g$ is a homotopy, then the subdivision argument shows $\tilde\eta([f])=\tilde\eta([g])$.
    Precisely, this means we subdivide $I\times I$ into sufficiently small subsquares, and by Lebesgue's lemma, each square falls into a single $U\in\mathscr{O}$.
    Then $[f]=[g]$ in $\Pi(X)$ is a consequence of a finite number of relations, each of which holds in one of the $\Pi(U)$.
    Therefore $\tilde\eta([f])=\tilde\eta([g])$.

    The above construction is clearly unique, hence this verifies the universal property and proves the theorem.
\end{proof}

The fundamental group version of van Kampen theorem (Theorem~\ref{van Kampen}) is proved by categorical nonsense.

\begin{lem}
    The van Kampen theorem (Theorem~\ref{van Kampen}) holds when the cover $\mathscr{O}$ is finite.
\end{lem}
\begin{proof}
    This is based on the categorical nonsense about skeleta of categories.
    Given a morphism between $\mathscr{O}$-shaped diagrams $\eta:\pi_1|_\mathscr{O}\to\underline{G}$, we must show that there is a unique homomorphism $\tilde\eta:\pi_1(X,x)\to G$ that restricts to $\eta_U$ on $\pi_1(U,x)$.

    Let $J_U:\pi_1(U,x)\to\Pi(U)$ be the inclusion functor, then $J_U$ is a categorical equivalence.
    The inverse is given by for $y\in X$, the path $c_x$ if $y=x$, so that $F_U\circ J_U=\operatorname{id}$; and a path lies entirely in $U$ as long as $y\in U$, this is possible since $\mathscr{O}$ is finite, we can take $U$ be the intersection of all open subsets in $\mathscr{O}$ that contains $y$.
    This ensures the chosen paths determine compatible inverse equivalence.
    Thus the functors
    $\begin{tikzcd}
        \pi(U)\ar[r, "{F_U}"] & \pi_1(U,x) \ar[r, "{\eta_U}"] & G
    \end{tikzcd}$
    specify an $\mathscr{O}$-shaped diagram of groupoids $\Pi|_\mathscr{O}\to\underline{G}$.
    By the fundamental groupoid version of van Kampen theorem, there is a unique map of groupoids
    $\xi:\Pi(X)\to G$
    that restricts to $\eta_U\circ F_U$ on $\Pi(U)$ for each $U$.
    Then the composite
    $\begin{tikzcd}
        \pi_1(X,x) \ar[r, "J"] & \Pi(X) \ar[r, "\xi"] & G
    \end{tikzcd}$
    is the required homomorphism $\tilde\eta$.
    By the commutativity of the diagram
    \[\begin{tikzcd}
        \ & \pi_1(X,x) \ar[ld, "{\operatorname{id}}"'] & \ \\
        \pi_1(X,x) \ar[r, "J"] & \Pi(X) \ar[u, "F"'] \ar[r, "\xi"] & G \\
        \pi_1(U,x) \ar[u, dashed] \ar[r, "{J_U}"] \ar[rr, bend right, "{\operatorname{id}}"] & \Pi(U) \ar[u] \ar[r, "{F_U}"] & \pi_1(U,x), \ar[u, "{\eta_U}"'] 
    \end{tikzcd}\]
    $\tilde\eta$ restricts to $\eta_U$ on $\pi_1(U,x)$ for $U\in\mathscr{O}$.
    It is unique because $\xi$ is unique.
    Hence we verified the universal property.
\end{proof}

\begin{proof}[Proof of van Kampen Theorem]
    Let $\mathscr{F}$ be the set of those finite subsets of the cover $\mathscr{O}$ that are closed under finite intersection.
    For $\mathscr{S}\in\mathscr{F}$, let $U_\mathscr{S}$ be the union of $U$'s in $\mathscr{S}$.
    Then $\mathscr{S}$ is a cover of $U_\mathscr{S}$ to which the lemma applies.
    Thus
    \[\colim_{U\in\mathscr{S}}\pi_1(U,x)\cong\pi_1(U_\mathscr{S},x).\]
    Regard $\mathscr{F}$ as a category with a morphism $\mathscr{S}\to\mathscr{T}$ whenever $U_\mathscr{S}\subset U_\mathscr{T}$.
    We claim first that
    \[\colim_{\mathscr{S}\in\mathscr{F}}\pi_1(U_\mathscr{S},x)\cong\pi_1(X,x).\]
    In fact, by the subdivision argument, any loop $I\to X$ and any homotopy equi\-valence $h:I\times I\to X$ between loops has image in some $U_\mathscr{S}$.
    This implies that $\pi_1(X,x)$, together with the homomorphisms $\pi_1(U_\mathscr{S},x)\to\pi_1(X,x)$, has the universal property that characterizes the claimed colimit.
    We claim next that
    \begin{align*}
        \colim_{U\in\mathscr{O}}\pi_1(U,x)&\cong\colim_{\mathscr{S}\in\mathscr{F}}\pi_1(U_\mathscr{S},x)\\
        &\cong\colim_{\mathscr{S}\in\mathscr{F}}\colim_{U\in\mathscr{S}}\pi_1(U,x),
    \end{align*}
    and this will complete the proof.
    By a comparison of universal properties, this iterated colimit is isomorphic to the single colimit
    \[\colim_{(U,\mathscr{S})\in(\mathscr{O},\mathscr{F})}\pi_1(U,x),\]
    here the indexing category $(\mathscr{O},\mathscr{F})$ has objects the pairs $(U,\mathscr{S})$ with $U\in\mathscr{S}$; a morphism $(U,\mathscr{S})\to(V,\mathscr{T})$ whenever both $U\subset V$ and $U_\mathscr{S}\subset U_\mathscr{T}$.
    We show the isomorphism.
    We first give a general construction.
    For categories $\mathsf{C}\xrightarrow{F}\mathsf{D}\xrightarrow{G}\mathsf{E}$ with $\mathsf{E}$ cocomplete, we construct a natural morphism $F_*:\colim_{c\in\mathsf{C}}G\circ F(c)\to\colim_{d\in\mathsf{D}}G(d)$.
    Let $c\in\mathsf{C}$, then $F(c)\in\mathsf{D}$, we have a natural morphism $G\circ F(c)=G(F(c))\to\colim_{d\in\mathsf{D}}G(d)$, and then by universal property, there is a unique $F_*:\colim_{c\in\mathsf{C}}G\circ F(c)\to\colim_{d\in\mathsf{D}}G(d)$.
    Now let functors $i:\mathscr{O}\to(\mathscr{O},\mathscr{F})$, $U\mapsto(U,\{U\})$, and $p:(\mathscr{O},\mathscr{F})\to\mathscr{O}$, $(U,\mathscr{S})\to U$.
    We omit the notation for base point, and the required isomorphism becomes
    \[\colim_{U\in\mathscr{O}}\pi_1(U)\cong\colim_{(U,\mathscr{S})\in(\mathscr{O},\mathscr{F})}\pi_1\circ p(U,\mathscr{S}).\]
    Notice that $p\circ i=\operatorname{id}$, by the composite of functors
    \[\mathscr{O}\xrightarrow{i}(\mathscr{O},\mathscr{F})\xrightarrow{p}\mathscr{O}\xrightarrow{\pi_1}\mathsf{Grp},\]
    we have a commutative diagram
    \[\begin{tikzcd}
        \pi_1\circ p\circ i(U) \ar[r, equal] \ar[d] & \pi_1\circ p(U,\{U\})=\pi_1\circ p(U,\mathscr{S}) \ar[r, equal] \ar[d] & \pi_1(U) \ar[d] \\
        \colim\limits_{U\in\mathscr{O}}\pi_1\circ p\circ i(U)\ar[r, "{i_*}"] \ar[rr, dashed, bend right] & \colim\limits_{(U,\mathscr{S})\in(\mathscr{O},\mathscr{F})}\pi_1\circ p(U,\mathscr{S}) \ar[r, "{p_*}"] & \colim\limits_{U\in\mathscr{O}}\pi_1(U).
    \end{tikzcd}\]
    The right arrows in second row are homomorphisms induced by functor $i$ and $p$, hence the downward arrows are all natural homomorphisms.
    But since $p\circ i=\operatorname{id}$, $\colim_{U\in\mathscr{O}}\pi_1\circ p\circ i(U)=\colim_{U\in\mathscr{O}}\pi_1(U)$, the dashed arrow must be identity.
    Similarly, we have commutative diagram
    \[\begin{tikzcd}
        \pi_1\circ p\circ i\circ p(U,\mathscr{S}) \ar[r, equal] \ar[d] & \pi_1\circ p\circ i(U) \ar[r, equal] \ar[d] & \pi_1\circ p(U,\{U\})=\pi_1\circ p(U,\mathscr{S}) \ar[d]\\
        \colim\limits_{(U,\mathscr{S})\in(\mathscr{O},\mathscr{F})}\pi_1\circ p\circ i\circ p(U,\mathscr{S}) \ar[r, "p_*"] \ar[rr, bend right, dashed] & \colim\limits_{U\in\mathscr{O}}\pi_1\circ p\circ i(U) \ar[r, "i_*"] & \colim\limits_{(U,\mathscr{S})\in(\mathscr{O},\mathscr{F})}\pi_1\circ p(U,\mathscr{S}),
    \end{tikzcd}\]
    and the dashed homomorphism is identity.
    This shows $\colim_{U\in\mathscr{O}}\pi_1(U)\cong\colim_{(U,\mathscr{S})\in(\mathscr{O},\mathscr{F})}\pi_1\circ p(U,\mathscr{S})$.
\end{proof}

\end{document}